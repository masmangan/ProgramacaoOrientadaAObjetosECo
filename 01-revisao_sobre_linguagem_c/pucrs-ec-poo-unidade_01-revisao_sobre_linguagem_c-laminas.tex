\documentclass[aspectratio=169]{beamer}
\usepackage[utf8]{inputenc}
\usepackage[T1]{fontenc}
\usepackage[brazil]{babel}
\usepackage{ragged2e}
\usepackage{booktabs}
\usepackage{verbatim}
\usetheme{AnnArbor}
\usecolortheme{orchid}
\usefonttheme[onlymath]{serif}

\newcommand\setItemnumber[1]{\setcounter{enumi}{\numexpr#1-1\relax}}

\AtBeginSection[]{
  \begin{frame}
  \vfill
  \centering
  \begin{beamercolorbox}[sep=8pt,center,shadow=true,rounded=true]{title}
    \usebeamerfont{title}\insertsectionhead\par%
  \end{beamercolorbox}
  \vfill
  \end{frame}
}

\title[\sc{Revisão sobre Linguagem C}]{Revisão sobre Linguagem C}
\author[Roland Teodorowitsch]{Roland Teodorowitsch}
%\institute[LP2 - EC - PUCRS]{Laboratório de Programação II - Curso de Engenharia de Computação - PUCRS}
\institute[POO - EC - PUCRS]{Programação Orientada a Objetos - ECo - Curso de Engenharia de Computação - PUCRS}
\date{4 de março de 2021}

\begin{document}
\justifying

%-------------------------------------------------------
\begin{frame}
	\titlepage
\end{frame}

%=======================================================
\section{Questões}

%-------------------------------------------------------
\begin{frame}\frametitle{Questão 1}
\begin{enumerate}
\setItemnumber{1}
\item Considere as seguintes afirmações:
	\begin{enumerate}[I.]
	\item Linguagens de Programação de \_\_\_\_\_\_\_\_ apresentam um nível de abstração mais alto, sem exigir conhecimento do código de máquina.
	\item Linguagens de Programação de \_\_\_\_\_\_\_\_ podem ser ou \_\_\_\_\_\_\_\_ (quando se gera código executável e erros de sintaxe são detectados antes da execução) ou \_\_\_\_\_\_\_\_ (quando instruções são decodificadas e executadas durante a própria execução).
	\item Linguagens \_\_\_\_\_\_\_\_ usam códigos mnemônicos, que possuem uma correspondência direta com as instruções em linguagem de máquina.
	\end{enumerate}

Qual das alternativas abaixo completa correspondentemente e corretamente cada uma das afirmações acima?
	\begin{enumerate}[(A)]
	\item Alto Nível -- Alto Nível -- virtualizadas -- pré-compiladas -- de alto nível.
	\item Baixo Nível -- Alto Nível -- interpretadas -- compiladas -- de bancos de dados.
	\item Baixo Nível -- Baixo Nível -- interpretadas -- compiladas -- de baixo nível.
	\item Alto Nível -- Alto Nível -- compiladas -- interpretadas -- de montagem.
	\item Médio Nível -- Baixo Nível -- compiladas -- interpretadas -- virtualizadas.
	\end{enumerate}
\end{enumerate}
\end{frame}

%-------------------------------------------------------
\begin{frame}\frametitle{Questão 2}
\begin{enumerate}
\setItemnumber{2}
\item Escreva uma função recursiva em C para calcular o número de Fibonacci de ordem n. Lembre-se de que o número de Fibonacci de ordem n pode ser definido como a soma dos dois números de Fibonacci anteriores:\\
\emph{\texttt{Fibonacci ( n )  = Fibonacci(n-1) + Fibonacci(n-2)}}\\
e que \emph{\texttt{Fibonacci(1)=0}} e \emph{\texttt{Fibonacci(2)=1}}.\\
Por exemplo, para n igual a 5 o valor de Fibonacci correspondente será 3:
\begin{itemize}
\item \emph{\texttt{Fibonacci(1) = 0}}
\item \emph{\texttt{Fibonacci(2) = 1}}
\item \emph{\texttt{Fibonacci(3) = Fibonacci(1)+Fibonacci(2) = 0 + 1 = 1}}
\item \emph{\texttt{Fibonacci(4) = Fibonacci(2)+Fibonacci(3) = 1 + 1 = 2}}
\item \emph{\texttt{Fibonacci(5) = Fibonacci(3)+Fibonacci(4) = 1 + 2 = 3}}
\end{itemize}
\end{enumerate}
\end{frame}

%-------------------------------------------------------
\begin{frame}\frametitle{Questão 3}
\begin{enumerate}
\setItemnumber{3}
\item Considere a função criada na questão anterior e crie um programa que leia valores inteiros enquanto estes valores forem maiores ou iguais a um, calculando, para cada valor maior ou igual a um, o número de Fibonacci correspondente. Contabilize também quantos dos números de Fibonacci calculados eram pares, quantos eram ímpares e quantas vezes o resultado foi zero. O número de pares, ímpares e zeros deve ser impresso antes do final da execução do programa.
\end{enumerate}
\end{frame}

%-------------------------------------------------------
\begin{frame}\frametitle{Questão 4}
\begin{enumerate}
\setItemnumber{4}
\item Escreva um programa em C que leia dois vetores A e B, parcialmente preenchidos, cada um com no máximo 100 elementos. Inicialmente deve-se ler o número de elementos do vetor A (0 a 100), depois cada um dos valores do vetor A. A seguir deve-se ler o número elementos do vetor B (0 a 100), depois cada um dos valores do vetor B. Depois de ler os dois vetores, seu programa deve verificar quantas vezes o vetor A aparece dentro do vetor B, imprimindo este valor. Por exemplo, se tivéssemos A = \{ 3, 4 \} e B = \{ 1, 2, 3, 4, 1, 2, 3, 3, 4, 5 \}, o resultado seria 2.
\end{enumerate}
\end{frame}

%-------------------------------------------------------
\begin{frame}\frametitle{Questão 5}
{\scriptsize
\begin{enumerate}
\setItemnumber{5}
\item Sua tarefa é escrever trechos de programa em C para gerenciar itens de uma agenda de contatos. Para cada item da agenda será necessário armazenar: nome (até 40 caracteres úteis), telefone (no formato ``(99)99999-9999''), dia de nascimento, mês de nascimento e ano de nascimento. Desta forma:
\begin{enumerate}[a)]
\scriptsize
\item Declare um tipo correspondente ao registro (struct) para armazenar as informações dos contatos da agenda;
\item Escreva um trecho de programa que leia o número de contatos da agenda, aloque área dinamicamente para os contatos (usando como base o registro declarado no item a) e a seguir leia as informações de cada um dos contatos da agenda, armazenando essas informações na estrutura de dados alocada;
\item Escreva um trecho de programa para ler um dia e um mês, mostrando todos os contatos da agenda que fazem aniversário neste dia deste mês (levando em consideração o registro declarado no item a, bem como a lista de contatos alocada e lida no item b -- declare apenas novas variáveis necessárias para atender o que se pede neste item;
\item Escreva um trecho de programa que salve todos os dados lidos no item b em um arquivo chamado ``agenda.csv'' no formato CSV (\emph{Comma-Separated-Values}). Neste tipo de arquivo cada linha contém os dados de um contato, com cada campo delimitado por um separador (use ``;'' como separador). Por exemplo, se ``João da Silva'', de telefone ``(51)99123-4567'', nascido no dia 13 do mês 8 de 1988 é um contato, então a linha gerada no arquivo CSV para este contato seria:\\
\texttt{João da Silva;(51)99123-4567;13;8;1988}\\
Da mesma forma, declare apenas novas variáveis necessárias para atender o que se pede neste item.
\item Escreva o trecho de programa para desalocar a estrutura de dados alocada no item b.
\end{enumerate}
\end{enumerate}
}
\end{frame}

%-------------------------------------------------------
\begin{frame}\frametitle{Questão 6}
\begin{enumerate}
\setItemnumber{6}
\item Um programa em C pode receber informações da linha de comandos através da especificação de dois parâmetros na função main(): \texttt{argc} (valor inteiro que indica o número de palavras colocado na linha de comandos) e \texttt{argv} (vetor de ponteiros de caracteres apontando para o texto de cada uma destas palavras). Escreva um programa em C que recebe uma lista de valores inteiros como argumentos da linha de comandos e imprime o produtório destes valores. Por exemplo, o programa questao.c, poderia ser executado da seguinte forma:\\
\texttt{\$ questao 1 2 2 3\\
RESULTADO = 12}\\
Lembre-se de que o ``primeiro argumento'' é sempre o nome do executável e que todos os argumentos correspondem a vetores de caracteres. Pode-se converter um vetor de caracteres para um valor inteiro usando a função \texttt{int atoi(char *nptr);} definida em \texttt{stdlib.h}.

\end{enumerate}
\end{frame}

%-------------------------------------------------------
\end{document}

