\documentclass[aspectratio=169]{beamer}
\usepackage[utf8]{inputenc}
\usepackage[T1]{fontenc}
\usepackage[brazil]{babel}
\usepackage{ragged2e}
\usepackage{booktabs}
\usepackage{verbatim}
\usepackage{gensymb}
\usepackage{multirow}
\usepackage{xcolor,colortbl}
\definecolor{verde}{rgb}{0,0.5,0}
\usepackage{listings}

\lstset{language=C++,
	backgroundcolor=\color{green!10},
	basicstyle=\ttfamily,
	keywordstyle=\color{blue}\ttfamily,
	stringstyle=\color{red}\ttfamily,
	commentstyle=\color{green}\ttfamily,
	morecomment=[l][\color{magenta}]{\#},
	literate=
	{á}{{\'a}}1
	{à}{{\`a}}1
	{ã}{{\~a}}1
	{â}{{\^a}}1
	{é}{{\'e}}1
	{ê}{{\^e}}1
	{í}{{\'i}}1
	{ó}{{\'o}}1
	{õ}{{\~o}}1
	{ú}{{\'u}}1
	{ü}{{\"u}}1
	{ç}{{\c{c}}}1
	{Á}{{\'A}}1
	{À}{{\`A}}1
	{Ã}{{\~A}}1
	{Â}{{\^A}}1
	{É}{{\'E}}1
	{Ê}{{\^E}}1
	{Í}{{\'I}}1
	{Ó}{{\'O}}1
	{Õ}{{\~O}}1
	{Ú}{{\'U}}1
	{Ü}{{\"U}}1
	{Ç}{{\c{C}}}1
}

\usetheme{AnnArbor}
\usecolortheme{orchid}
\usefonttheme[onlymath]{serif}

\AtBeginSection[]{
  \begin{frame}
  \vfill
  \centering
  \begin{beamercolorbox}[sep=8pt,center,shadow=true,rounded=true]{title}
    \usebeamerfont{title}\insertsectionhead\par%
  \end{beamercolorbox}
  \vfill
  \end{frame}
}

\title[\sc{Ponteiros e Alocação Dinâmica}]{Ponteiros e Alocação Dinâmica}
\author[Roland Teodorowitsch]{Roland Teodorowitsch}
%\institute[LP2 - EC - PUCRS]{Laboratório de Programação II - Curso de Engenharia de Computação - PUCRS}
\institute[POO - EC - PUCRS]{Programação Orientada a Objetos - ECo - Curso de Engenharia de Computação - PUCRS}
\date{11 de outubro de 2023}

\begin{document}
\justifying

%-------------------------------------------------------
\begin{frame}
	\titlepage
\end{frame}

%=======================================================
\section{Ponteiros em C++}

%-------------------------------------------------------
\begin{frame}[fragile]\frametitle{Revisão}
\begin{itemize}
	\item O operador \texttt{*} antecedendo o nome de uma variável em uma declaração, define um ponteiro o respectivo tipo
\begin{lstlisting}[language=C++]
double *mediaPtr;
\end{lstlisting}
	\item Operador de referência (\texttt{\&}): do lado esquerdo de uma variável, permite recuperar o endereço desta variável (pode ser lido como ``endereço de'')
\begin{lstlisting}[language=C++]
mediaPtr = &media;
\end{lstlisting}
	\item Operador de dereferência (\texttt{*}): quando utilizado sozinho, à esquerda de um ponteiro, retorna o conteúdo do ponteiro (pode ser lido como ``conteúdo ou valor apontado por'')
\begin{lstlisting}[language=C++]
cout << "media= " << *mediaPtr << endl;
\end{lstlisting}
\end{itemize}
\end{frame}

%-------------------------------------------------------
\begin{frame}[fragile]\frametitle{Exemplo: \texttt{exemplo01.cpp}}
\lstinputlisting{src/exemplo01.cpp}
\end{frame}

%-------------------------------------------------------
\begin{frame}[fragile]\frametitle{Inicialização de ponteiros}
\begin{itemize}
	\item É uma boa prática de programação em C/C++ sempre inicializar tanto variáveis quanto ponteiros
	\item Em Java, o compilador inicializa as variáveis automaticamente com 0 (zero), mas em C/C++ isso NÃO ocorre
	\item Em C++, para ponteiros pode-se usar \texttt{nullptr}:
\begin{lstlisting}[language=C++]
int *dados = nullptr;
\end{lstlisting}
	\item Para uma discussão sobre o uso de 0 ou NULL em C++, veja: \url{http://www.cplusplus.com/forum/beginner/5604/}
\end{itemize}
\end{frame}

%-------------------------------------------------------
\begin{frame}[fragile]\frametitle{Impressão de ponteiros}
\begin{itemize}
	\item Em C++, ponteiros podem ser impressos usando \texttt{cout} normalmente:
\lstinputlisting{src/exemplo02.cpp}
\end{itemize}
\end{frame}

%-------------------------------------------------------
\begin{frame}[fragile]\frametitle{Vetores}
\begin{itemize}
	\item Em C/C++, ``um vetor é um ponteiro para determinado tipo de dado'':
\lstinputlisting{src/exemplo03.cpp}
\end{itemize}
\end{frame}

%-------------------------------------------------------
\begin{frame}[fragile]\frametitle{Aritmética de Ponteiros}
\begin{itemize}
	\item É possível realizar operações sobre ponteiros
	\item Somar um valor inteiro a um ponteiro de determinado tipo significa ``deslocar'' a referência
	\item O deslocamento será igual a \texttt{sizeof(tipo)*valorInteiro}
\lstinputlisting{src/exemplo04.cpp}
\end{itemize}
\end{frame}

%-------------------------------------------------------
\begin{frame}[fragile]\frametitle{Ponteiros de Objetos}
\lstinputlisting[basicstyle=\ttfamily\scriptsize]{src/exemplo05.cpp}
\end{frame}

%=======================================================
\section{Alocação Dinâmica em C++}

%-------------------------------------------------------
\begin{frame}[fragile]\frametitle{Alocação Dinâmica (Revisão)}
\begin{itemize}
	\item Alocação Estática
\begin{lstlisting}[language=C++]
int num;
char v[10];
\end{lstlisting}
	\item Alocação Dinâmica: em C++ é feita com os operadores \texttt{new} (para alocar) e \texttt{delete} (para desalocar)
\begin{lstlisting}[language=C++]
int *ptr_i = new int;
MinhaClasse *meuObjeto = new MinhaClasse;
// ...
delete meuObjeto;
delete ptr_i;
\end{lstlisting}
\end{itemize}
\end{frame}

%-------------------------------------------------------
\begin{frame}[fragile]\frametitle{Alocação Dinâmica de Vetores}
\begin{itemize}
	\item No exemplo anterior, foi alocado um espaço para armazenar apenas um dado
	\item Para alocar um vetor, basta especificar o tamanho desse vetor no momento da alocação
	\item O acesso é feito exatamente como se fosse um vetor
\lstinputlisting[basicstyle=\ttfamily\scriptsize]{src/exemplo04.cpp}
	\end{itemize}
\end{frame}

%-------------------------------------------------------
\begin{frame}[fragile]\frametitle{Alocação Dinâmica de Matrizes}
\lstinputlisting[basicstyle=\ttfamily\tiny]{src/matriz_dinamica.cpp}
\end{frame}

%-------------------------------------------------------
\begin{frame}[fragile]\frametitle{Realocação de Memória}
\begin{itemize}
	\item Se for preciso aumentar o tamanho de um vetor alocado dinamicamente, pode-se usar:\\\texttt{realloc(ponteiro, novo\_tamanho)}
	\item Exemplo:
\begin{lstlisting}[language=C++]
// A area de memoria eh realocada, e todos os dados sao
// copiados da area antiga para a area nova
cadastro = (CPessoa *)realloc(cadastro, 20);
\end{lstlisting}
\end{itemize}
\end{frame}

%=======================================================
\section{Lista de Exercícios}

%-------------------------------------------------------
\begin{frame}[fragile]\frametitle{Exercício 1}
\begin{enumerate}
	\small	
	\item Analise o programa abaixo, considerando que as variáveis globais \texttt{a} (\texttt{int}, 4 \emph{bytes}), \texttt{b} (\texttt{int}, 4 \emph{bytes}), \texttt{ptr} (ponteiro para \texttt{int}, 8 \emph{bytes}) e \texttt{v} (vetor com 4 elementos \texttt{int}, 16 \emph{bytes}) tenham sido alocadas nos endereços especificados nos respectivos comentários, ou seja, respectivamente, nos endereços \texttt{0xf180}, \texttt{0xf184}, \texttt{0xf188} e \texttt{0xf190}. A partir destas informações, mostre a saída do programa.
\fontsize{5pt}{5pt}\selectfont{
\lstinputlisting{src/exercicio01.cpp}
}
	\end{enumerate}
\end{frame}

%-------------------------------------------------------
\begin{frame}[fragile]\frametitle{Exercício 2}
\begin{enumerate}
	\setcounter{enumi}{1}
	\item Analise a classe do programa abaixo e crie os métodos necessários para o armazenamento dos dados digitados pelo usuário.
\lstinputlisting[basicstyle=\ttfamily\tiny]{src/exercicio02-template.cpp}
\end{enumerate}
\end{frame}

%-------------------------------------------------------
\begin{frame}\frametitle{Exercícios 3-6}
\begin{enumerate}
	\small
	\setcounter{enumi}{2}
	\item Escreva um programa em C++ que recebe um vetor de números inteiros. Esse vetor de 10 posições deve
ser criado de forma dinâmica e inicializado de 1 a 10. Ao final, deve-se imprimir na tela usando
aritmética de ponteiros.
	\item Crie duas funções em C++, uma para imprimir um vetor de inteiros e outra para imprimir um vetor de valores reais (\texttt{double}). Ambas as funções devem receber um ponteiro para o vetor e o tamanho do vetor, e devem usar apenas a aritmética de ponteiros.
	\item Escreva um programa em C++ que inicialize um vetor de 100 números reais (\texttt{double}) com valores aleatórios quaisquer, e depois calcule a média apenas dos números que se encontram nos índices pares (iniciando em 0) desse vetor. Utilize apenas aritmética de ponteiros na sua implementação.
	\item Modifique o programa do exercício anterior de forma que ele receba o tipo de média a ser calculada como parâmetro da linha de comandos: se for passado ``0'', faça a média sobre os índices pares; se for passado ``1'', faça a média sobre os índices ímpares; e, caso seja outro número, faça a média de todos os números do vetor.	
\end{enumerate}
\end{frame}

%=======================================================
\section{Créditos}

%-------------------------------------------------------
\begin{frame}\frametitle{Créditos}
\begin{itemize}
	\item Estas lâminas contêm trechos de materiais disponibilizados pelos professores Rafael Garibotti, Márcio Sarroglia Pinho, Marco Mangan, Matheus Trevisan e Edson Moreno.
\end{itemize}
\end{frame}

%=======================================================
\section{Soluções}

%-------------------------------------------------------
\begin{frame}[fragile]\frametitle{Exercício 1}
\begin{lstlisting}[language={}]
Memória:                         Saída do programa:
========                         ==================
0xf180 | a    | 4                0xf19c
0xf184 | b    | 4                9
0xf188 | ptr  | 0xf190           4
0xf190 | v[0] | 4                3
0xf194 | v[1] | 3                9
0xf198 | v[2] | 9                4
0xf19c | v[3] | 4
\end{lstlisting}
\end{frame}

%-------------------------------------------------------
\begin{frame}[fragile]\frametitle{Exercício 2: \texttt{exercicio02.cpp}}
\fontsize{5pt}{5pt}\selectfont{
\lstinputlisting{src/exercicio02.cpp}
}
\end{frame}

%-------------------------------------------------------
\begin{frame}[fragile]\frametitle{Exercício 3: \texttt{exercicio03.cpp}}
\lstinputlisting[basicstyle=\ttfamily\tiny]{src/exercicio03.cpp}
\end{frame}

%-------------------------------------------------------
\begin{frame}[fragile]\frametitle{Exercício 4: \texttt{exercicio04.cpp}}
\lstinputlisting[basicstyle=\ttfamily\tiny]{src/exercicio04.cpp}
\end{frame}

%-------------------------------------------------------
\begin{frame}[fragile]\frametitle{Exercício 5: \texttt{exercicio05.cpp}}
\lstinputlisting[basicstyle=\ttfamily\tiny]{src/exercicio05.cpp}
\end{frame}

%-------------------------------------------------------
\begin{frame}[fragile]\frametitle{Exercício 6: \texttt{exercicio06.cpp}}
\lstinputlisting[basicstyle=\ttfamily\tiny]{src/exercicio06.cpp}
\end{frame}

%-------------------------------------------------------
\end{document}

